% !TeX root = main.tex
%%%%%%%%%% SVN Info %%%%%%%%%%%%%%%%%%%%%%%%%%%%%%%%%%%%%%%%
%% $Date: 2020-02-05 16:17:49 +0100 (Mi, 05 Feb 2020) $
%% $Author: cmueller $
%% $Rev: 5 $
%%%%%%%%%%%%%%%%%%%%%%%%%%%%%%%%%%%%%%%%%%%%%%%%%%%%%%%%%%%%

%%%% Meta data for title page;
%%%% Please do not change anything below this line
%%%% ------------------------
\makeatletter
%% The pursued degree, defaults to Bachelor
\newcommand*{\degree}[1]{\gdef\@degree{#1}%
}
\newcommand*{\@degree}{Bachelor}
%% The degree type, defaults to Science
\newcommand*{\degreeType}[1]{\gdef\@degreeType{#1}%
}
\newcommand*{\@degreeType}{Science}
%% The degree program, defaults to Business Informatics
\newcommand*{\degreeProgram}[1]{\gdef\@degreeProgram{#1}%
}
\newcommand*{\@degreeProgram}{Business Informatics}
%% The type of Thesis, uses degreeType
\newcommand*{\thesisType}[1]{\gdef\@thesisType{#1}%
}
\newcommand*{\@thesisType}{\@degree 's Thesis}
%% The supervisor, prints text if not set.
\newcommand*{\supervisor}[1]{\gdef\@supervisor{#1}%
}
\newcommand*{\@supervisor}{\texttt{\textbackslash supervisor} currently not set.}
%% The (first) advisor, prints text if not set.
\newcommand*{\advisorA}[1]{\gdef\@advisorA{#1}%
}
\newcommand*{\@advisorA}{Please set \texttt{\textbackslash advisorA} if there is only one advisor.}
%% The (second) advisor, simply blank if not set.
\newcommand*{\advisorB}[1]{\gdef\@advisorB{#1}%
}
\newcommand*{\@advisorB}{}
%% The matriculation number, defaults to 0123456.
\newcommand*{\matriNo}[1]{\gdef\@matriNo{#1}%
}
\newcommand*{\@matriNo}{0123456}
%% The keywords, defaults to empty.
\newcommand*{\keywords}[1]{\gdef\@keywords{#1}%
}
\newcommand*{\@keywords}{}
\makeatother
%%%% ------------------------
%%%% Please do not change anything above this line


%%%% Names
\newcommand{\Channel}{Link}
\newcommand{\channel}{\expandafter\lowercase\expandafter{\Channel}}

\newcommand{\outSign}{\mathrm{out}}
\newcommand{\inSign}{\mathrm{in}}

%%%% Sets
\newcommand{\parties}{\mathcal{P}}                 %% The set of all parties P
\newcommand{\party}[2][]{{P}_{#2}^{#1}}            %% A single Party p
\newcommand{\tape}[2]{{T}_{#1}^{#2}}               %% A tape
\newcommand{\inTape}[2][]{%
  \ifthenelse{\isempty{#1}}%
    {\tape{#2}{\inSign}}%
    {\tape{#2}{#1, \inSign}}}                      %% A single input tape
\newcommand{\outTape}[2][]{%
  \ifthenelse{\isempty{#1}}%
    {\tape{#2}{\outSign}}%
    {\tape{#2}{#1, \outSign}}}                     %% A single output tape
\newcommand{\dataSymbols}{\Sigma}                  %% The data alphabet
\newcommand{\tapeSymbols}{\Gamma}                  %% The tape alphabet
\newcommand{\N}{\mathbb{N}}                        %% The natural numbers
\newcommand{\radioParSet}[1][]{\Phi^{#1}}                  %% All possible signal parameters to choose from
\newcommand{\radioParSetIn}{\radioParSet[\inSign]}    %% 
\newcommand{\radioParSetOut}{\radioParSet[\outSign]}  %% 


%%%% Parameters
%\newcommand{\maxTapes}{n}                          %% The maximum number of tapes, e.g., channels
\newcommand{\timeslot}[1]{{t}_{#1}}                %% Symbol for atomic unit of time
\newcommand{\curTime}{\timeslot{\mathrm{cur}}}     %% The current time slot (epochs)
\newcommand{\delay}[2][]{{d}_{#2}^{#1}}            %% Propagation delay
\newcommand{\initial}{\varepsilon}                      %% Initial symbol for tape cells
\newcommand{\collide}{\textrm{\P}}                 %% Collision/jammed symbol
\newcommand{\dataBit}{\mathsf{b}}                  %% A symbol from the data alphabet
\newcommand{\tapeSymbol}{\mathsf{\theta}}          %% A symbol from the tape alphabet
\newcommand{\sleepTime}{\tau}                      %% The random jammer's sleep time
\newcommand{\multSymbols}{\nu}                     %% The amount of symbols per jam period
\newcommand{\radioPar}{\varphi}                    %% Represents (radio/energy) signal parameters, e.g. RSSI/LQI
\newcommand{\radioParDef}{\varphi_{\mathsf{def}}}  %% Represents standard/default value for signal param
\newcommand{\dataTuple}{\mathsf{\beta}}            %% A tuple of data bit and transmission param
%\newcommand{\aChannelOpt}[2][i]{\left( \inTape[#2]{#1}, \delay[#2]{#1}, f^{#2}_{#1} \right)}
%\newcommand{\aChannel}[1][i]{\aChannelOpt[#1]{}}
\newcommand{\aChannelOpt}[2][i]{\left( \tape{#1}{#2}, \delay[#2]{#1}, f^{#2}_{#1} \right)}
\newcommand{\aChannel}[1][i]{\aChannelOpt[#1]{}}
\newcommand{\msg}{\mathsf{m}}                      %% A message
\newcommand{\lmax}{\ensuremath{l_{\max}}}          %% Maximum message length

%%%% Functions
%% Typesetting
\DeclareMathOperator{\Tapes}{\mathsf{Tapes}}       %% Denotes (unspecified) tapes
\DeclareMathOperator{\In}{\mathsf{InTapes}}             %% Denotes input tapes
\DeclareMathOperator{\Out}{\mathsf{OutTapes}}           %% Denotes output tapes
\DeclareMathOperator{\Chn}{\mathsf{Lnk}}           %% Denotes connection from out- to input tape
\DeclareMathOperator{\Read}{\mathsf{Read}}         %% Denotes the read function
\DeclareMathOperator{\Write}{\mathsf{Write}}       %% Denotes the write function
\DeclareMathOperator{\cond}{\mathsf{cond}}         %% Denotes the selective jammer's condition

%% Function definitions
\newcommand{\allTapes}[2][]{\Tapes ( \party[#1]{#2} ) }                %% Gives the set of all Tapes for Party p
\newcommand{\inTapes}[2][]{\In ( \party[#1]{#2} ) }                     %% Gives the set of all Input Tapes for Party p
\newcommand{\outTapes}[2][]{\Out ( \party[#1]{#2} ) }                   %% Gives the set of all Output Tapes for Party p
%\newcommand{\channels}[3][]{{\Chn}_{\party[#1]{#2}} ( \outTape[#1]{#3} ) }    %% Gives the set of all Channels for Party P and Output Tape T*
%\newcommand{\readTape}[3][]{{\Read}_{\party[#1]{#2}} ( \inTape[#1]{#3} ) }          %% Read from Party P's input Tape T
%\newcommand{\writeTape}[4][]{\Write_{\party[#1]{#2}} ( \outTape[#1]{#3}, {#4} ) }   %% Write to Party P's output Tape T
\newcommand{\channels}[3][]{{\Chn}_{\party[#1]{#2}} ( \tape{#3}{#1} ) }    %% Gives the set of all Channels for Party P and Tape T*
\newcommand{\readTape}[3][]{{\Read}_{\party[#1]{#2}} ( \tape{#3}{#1} ) }          %% Read from Party P's input Tape T
\newcommand{\writeTape}[4][]{\Write_{\party[#1]{#2}} ( \tape{#3}{#1}, {#4} ) }   %% Write to Party P's output Tape T
\newcommand{\jamCond}[1]{{\cond ( #1 )}}   %% Condition for selective jammer to become active

%%%% Misc.
\newcommand{\ie}{i.\,e., }
\newcommand{\eg}{e.\,g., }
\newcommand{\etal}{et\,al.}
\newcommand{\mitm}{\mathrm{mim}}
\newcommand{\eav}{\mathrm{eav}}
\newcommand{\adversary}{\mathrm{adv}}

%\newcommand{\noopsort}[1]{}     %% Fix some weird bibtex sorting issue  {\noopsort{Ieee}}
