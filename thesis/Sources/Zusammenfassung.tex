\begin{center}
  \textsc{Zusammenfassung}
\end{center}
%
\selectlanguage{ngerman}
\noindent
Diese Bachelorarbeit befasst sich mit der Implementierung des ChaCha Plugins für CrypTool 2.

\medskip
\noindent
Die Arbeit stellt die Familie der ChaCha-Chiffren vor; erklärt, wozu das Plugin in der Lage ist; und gibt Einblick in den Entwicklungsprozess des Plugins.

\medskip
\noindent
ChaCha wird seit 2014 im Transport Layer Security-Protokoll (TLS) verwendet und ist daher für die angewandte moderne Kryptographie sehr relevant. Aufgrund der Bedeutung von ChaCha sollte sein internes Design der breiten Öffentlichkeit zugänglicher gemacht werden. Dies ist das eigentliche Ziel des Plugins.

\medskip
\noindent
Das Ziel wird erreicht, indem man sich auf eine detaillierte, aber leicht verständliche Visualisierung des Verschlüsselungsprozesses konzentriert.
CrypTool 2 ist die beliebteste E-Learning-Plattform im Bereich der Kryptologie ist und wird in Schulen, Universitäten und Unternehmen eingesetzt. Durch die Integration des Plugins in CrypTool 2 wird so ein breites Publikum erreicht.

\selectlanguage{english}