% !TeX spellcheck = en_GB

%%%%%%%%%%%%%%%%%%%%%%%%%%%%%%%%%%%%%%%%%%%%%%%%%%%%%%%%%%%%%%%%%%%%%%%%
\chapter{Introduction}

Applications of cryptology, the science behind creating encryptions (cryptography) and breaking them (cryptanalysis), date back far into ancient times. The first known example of  cryptography, a substitution cipher to conceal a formula for pottery glaze, is from 3500 BC \cite{history}. Ever since, advancements in technology pushed the boundary for secure ciphers. Nowadays, in the Age of Information, the need for keeping sensitive information private is as important as ever before and will only get more important with widespread adoption of new technologies such as the Internet of Things. This is why research into new encryption standards has to continuously take place.

The ChaCha cipher family by Daniel J. Bernstein is the result of such research and was published in 2008 \cite{chachaspec}. Because AES-GCM does not perform very well on devices without hardware acceleration such as wearable or mobile devices, Google has started to replace it in the Transport Layer Security (TLS) cipher suite of its browser Chrome with ChaCha20 for symmetric encryption and Poly1305 for authentication in 2014. Additionally, ChaCha is by design immune to previous TLS attacks such as padding-oracle or timing attacks and thus should improve security of HTTPS connections \cite{googlesecurityblog}.

This usage in TLS makes the ChaCha cipher family very attractive to include it in \textit{CrypTool 2} (CT2), a free open-source e-learning platform for cryptologists and those who aspire to be. It uses visual programming to teach cryptographic concepts and includes visualizations of many different ciphers, attacks and more. Thanks to its underlying architecture, one can easily write plug-ins for it using C\#, WPF and XAML. In fact, this was already done multiple times before by students as part of their bachelor thesis.

This thesis will describe the implemented ChaCha plug-in which includes an in-depth visualization of its internals. It also makes it possible to study the diffusion property of the cipher by letting the user input secondary values. The development process and underlying architecture will also be outlined to understand the reasoning behind some decisions.