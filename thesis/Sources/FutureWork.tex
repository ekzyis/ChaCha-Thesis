%%%%%%%%%%%%%%%%%%%%%%%%%%%%%%%%%%%%%%%%%%%%%%%%%%%%%%%%%%%%%%%%%%%%%%%%
\chapter{Future Work}
\label{chap:futureWork}

In this chapter, I will describe how the plugin could be further improved to make it more useful for the audience of CrypTool 2.

\begin{description}[style=nextline]
\item[Diffusion: XOR input and button to toggle between XOR and actual values] 

Adding a third input to the Diffusion page where the user can explicitly set the XOR of the primary and secondary value would better fit the needs of cryptanalysist who want to study the diffusion property of the cipher since one is more interested in the difference between two values than in their actual values. For example, if we just want to flip one bit, we could just type in a single 1 anywhere into the XOR input. \\
Using the same argument, a button with which one toggle between seeing the XOR of the intermediate values and their actual values during visualization would be useful.

\item[Automatic navigation]

As mentioned in Section \ref{sec:aesVisualization}, in the AES visualization, I have seen a button which lets the visualization run without further user interaction needed. A slider was provided to adjust the speed. I think that such a button could be useful for the ChaCha visualization, too, especially for the page about the ChaCha hash function with its many actions. \\
Since I am already using asynchronous navigation for the action slider (which I think would be needed to let a timer run without blocking the UI), I think implementing this feature could easily be integrated within the existing navigation system. Only page switches could maybe get tricky since the action navigation is handled within each page thus navigating out of a page may not be straight-forward. I have noticed that the automatic navigation in the AES visualization does stop between each step so switching the page automatically may not even be desired.

\pagebreak

\item[Salsa20 visualization]

As mentioned in Section \ref{sec:salsaCT2Plugin}, using the now existing codebase for ChaCha visualization to create a Salsa20 visualization would definitely increase the value gained from this thesis. This would at least need adaption of the XAML code for the state matrix initialization and the quarterround since ChaCha and Salsa20 differ in this aspects from each other. It would most likely even increase the value of the ChaCha visualization, since both ciphers could then be compared side-by-side. Since the visualizations would be very similar, comparing them and their diffusion property should be very easy.

\item[Localization and online help]

Currently, most texts are only localized in English. Only the memofields, component labels and plaintext value are also available in German. This should be changed in the near future. Further, there should be an online help for the ChaCha plugin which opens by selecting it and pressing F1.

\end{description}

%%%%%%%%%%%%%%%%%%%%%%%%%%%%%%%%%%%%%%%%%%%%%%%%%%%%%%%%%%%%%%%%%%%%%%%%